\chapter*{Introduction}         % ne pas numéroter
\label{chap:introduction}       % étiquette pour renvois
\phantomsection\addcontentsline{toc}{chapter}{\nameref{chap:introduction}} 


Les réseaux de neurones font parties de la branche de l'apprentissage supervisé qu'on nomme l'apprentissage profond. On dit « profond » puisqu'ils sont composées de plusieurs couches qui permettent de soutirer de l'information contenue dans des données. Avec la profondeur vient la complexité. Chaque neurone d'une couche sera lié avec chacun des neurones de la couche précédente et avec chacun des neurones de la couche suivante. Ainsi, rapidement les réseaux de neurones deviennent complexes et difficiles à interpréter. C'est dans ce contexte qu'on se questionne à savoir s'il est possible de les appliquer à un contexte actuariel classique, soit la modélisation de la fréquence de réclamation en assurance automobile à partir de données structurées. 

En effet, l'assurance en général requiert que les modèles de prévisions soient interprétables pour répondre aux conditions des régulateurs de marché. Les modèles classiques que sont les modèles linéaires généralisés permettent une interprétation claire. Par ce présent travail, on tente d'appliquer des méthodes d'interprétation de modèles de type « boîte noire » pour essayer de comprendre si les réseaux de neurones peuvent répondre aux conditions de régulations.

Tout d'abord, le \autoref{chap:RN} se consacre à la théorie des réseaux de neurones. On élabore brièvement leur fonctionnement. Pour une lecture en profondeur sur le sujet, on cite la référence en apprentissage profond \citep{Goodfellow-et-al-2016}. Pour une revue des réseaux de neurones avec l'emphase sur des problèmes actuariels, on cite \citet{denuit2019effective}. Plusieurs articles scientifiques sont consacrés à l'étude de problèmes actuariels à l'aide réseau de neurones. Dans \citet{wuthrich2019generalized}, on présente comment utilisé les réseaux de neurones pour améliorer les modèles actuariels classiques, article dont on se sert pour construire certains modèles. Plus particulièrement, on utilise les articles \citet{ferrario2018insights} et \citep{schelldorfer2019nesting}, qui font partie d'une série de tutoriels pour l'Association Suisse des Actuaires. On y explique comment utiliser les réseaux de neurones pour modéliser la fréquence de réclamation en assurance automobile et on y explique d'une façon plus appliquée l'approche CANN détaillée dans \citet{wuthrich2019generalized}. 

Le second chapitre est consacré à l'application de réseaux de neurones. On y détaille comment on construit les réseaux et on y compare les différents réseaux obtenus.