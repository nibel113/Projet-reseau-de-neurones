\chapter*{Conclusion}           % ne pas numéroter
\label{chap:conclusion}         % étiquette pour renvois
\phantomsection\addcontentsline{toc}{chapter}{\nameref{chap:conclusion}} % inclure dans TdM

La dernière sous-section a permis de montrer que le modèle CannDeep3 a bien plus de facilité à découvrir les interactions. La combinaison entre sa structure et la façon d'initialiser le réseau aura permis de trouver un modèle performant de façon efficace. De plus, les résultats obtenus l'ont été à partir d'hyperparamètres optimisés pour des réseaux différents. On peut alors penser qu'ils bénéficieraient d'une recherche plus approfondie. Une autre avenue qui pourrait être étudiée est l'initialisation. Un modèle de base plus performant que celui utilisé pourrait permettre d'initialiser le réseau d'une meilleure façon encore. 

Cependant, la base de données simple et le problème peu compliqué ne sont peut-être pas la meilleure façon de tester l'utilité des réseaux de neurones en actuariat. Une des avenues possibles est l'utilisation de données massives comme les données télématiques d'automobiles pour modéliser la fréquence de réclamation comme \citet{gao2019claims} l'ont fait.
